% {Details about what is expected in different parts of the proposal is stated in comments and within curly brackets like this one. You should delete them once your proposal is finalized and sent to faculty reviewer and examiner.}

\documentclass[10pt,a4paper]{proposal}

\usepackage{xcolor}
\usepackage{amssymb}
\usepackage{fullpage,url}
\usepackage{indentfirst}
\usepackage{amsmath}

\usepackage{fancyhdr}

\pagestyle{fancy}
\lhead{}
\chead{}
\rhead{\tiny{Joakim Olsson and Johan Liljegren}}
\lfoot{}
\cfoot{}
\rfoot{\tiny \thepage}
\renewcommand{\headrulewidth}{0pt}
\renewcommand{\footrulewidth}{0pt}

\title{Proposal for Bachelor Thesis in Software Engineering}
\begin{document}

\maketitle
\thispagestyle{fancy}


\section*{Base information}

Joakim Olsson laggmonkei@gmail.com

Johan Liljegren datanizze@gmail.com

Title (preliminary): Comparison between current open source Enterprise Service Busses

Academic Advisor:  

Start date: 17/02/12

Proposal ok by date: -

Presentation date: -


Thesis type: research\slash industrial


\section*{Background}

% {1 paragraph introducing and motivating the problem. Should answer: Which area of SE is this about? What particular part of that area? Why is this important?}

Enterprise Service Buses (ESB) \cite{falko07} are pivotal in software integration and are becoming extremely important in companies software solutions and  infrastructures, future and ancient alike. \cite{fenner03}
As such it is very important to accurately and repeatedly measure the performance of a variety of solutions available. 
There are currently several open source options available \cite{mehta11} and the current body of knowledge regarding their different performance aspects is severely lacking. 

% {2-3 paragraphs giving more detailed background. Should answer: What has been done by others in this area? What is our current knowledge?}

There has obviously been some measurement done by the producers of the ESBs however those are not very partial. Especially partial performance measurement are lacking, there are several reports made by the producers of some solutions. \cite{Perera07,mulevsjboss,mulevsglassfish,mulevsservicemix,mulesoft08} There has been some reasearch done in the area \cite{ESBthesis} however the number of ESB's tested are quite limited when compared to the most up to date list of popular ESBs \cite{mehta11}. Most importantly they do not explicitly declare what version they are using in their tests which makes their results hard to reproduce, judging by the dates on which their thesis was completed it is clear that all the ESBs that they use have recieved major upgrades which makes it even more important to produce new measurements.

% {1 paragraph detailing the gap in our current knowledge. Should answer: What is missing in our current knowledge? What is the main purpose of doing this project? Overall, what will we do?}
The main purpose of this thesis is to increase the knowledge of differences betwen modern open source ESBs. This wil hopefully lead to it being easier to further develop these software products and make it easier for end users to compare and evaluate based on its specific needs.


\section*{Aims and objectives}
% {1 sentence clearly stating the main aim of this thesis project.}
Compare several enterprise service busses in a controlled environment covering a multitude of tests 

% {3-7 objectives/sub-goals that follows from the main aim. By reaching them the main aim is fulfilled.}
\begin{itemize}
	\item Measure CPU usage
	\item Measure memory usage
	\item Usability during setup
	\item  Five ESB
\end{itemize}


\section*{Research questions}
% {3-7 clearly stated and answerable research questions.}
\begin{itemize}
	\item R1:
	\item R2:
	\item R3:
	\item R4:
\end{itemize}


\section*{Expected outcomes}
% {State the concrete results that will be the deliverables\slash output form the project. Should describe which form they are expected to take. Examples are: tables, models, guidelines, checklists, prototypes, designs, etc.}
pretty pretty graphs with numbers and shiny figures and stuff


\section*{Research Methodology}
% {Describe the research methodology that you will use in different parts of your project to be able to answer the research questions and produce the expected outcomes and thus fulfill the objectives and the main aim. Should clearly answer: How will you get an answer to each of the research questions stated above?}

	\subsection{Literature review}
		We will read different pappers and those that we consider relevant will be used too support the thesis.

	\subsection{Post-mortem analysis design}


\section*{Risks}
% {State the main threats to your study. Also state how you will overcome them.}

\footnotesize
\bibliographystyle{ieeetr}
\bibliography{refs}
% {7-12 key references that are relevant for your particular project. The reference list should be complete with all expected information for all entries.}
 
 \end{document}
