% !TEX TS-program = pdflatex
% !TEX encoding = UTF-8 Unicode

% This is a simple template for a LaTeX document using the "article" class.
% See "book", "report", "letter" for other types of document.

\documentclass[11pt]{article} % use larger type; default would be 10pt

\usepackage[utf8]{inputenc} % set input encoding (not needed with XeLaTeX)

%%% Examples of Article customizations
% These packages are optional, depending whether you want the features they provide.
% See the LaTeX Companion or other references for full information.

%%% PAGE DIMENSIONS
\usepackage{geometry} % to change the page dimensions
\geometry{a4paper} % or letterpaper (US) or a5paper or....
% \geometry{margin=2in} % for example, change the margins to 2 inches all round
% \geometry{landscape} % set up the page for landscape
%   read geometry.pdf for detailed page layout information

\usepackage{graphicx} % support the \includegraphics command and options

% \usepackage[parfill]{parskip} % Activate to begin paragraphs with an empty line rather than an indent

%%% PACKAGES
\usepackage{booktabs} % for much better looking tables
\usepackage{array} % for better arrays (eg matrices) in maths
\usepackage{paralist} % very flexible & customisable lists (eg. enumerate/itemize, etc.)
\usepackage{verbatim} % adds environment for commenting out blocks of text & for better verbatim
\usepackage{subfig} % make it possible to include more than one captioned figure/table in a single float
% These packages are all incorporated in the memoir class to one degree or another...

%%% HEADERS & FOOTERS
\usepackage{fancyhdr} % This should be set AFTER setting up the page geometry
\pagestyle{fancy} % options: empty , plain , fancy
\renewcommand{\headrulewidth}{0pt} % customise the layout...
\lhead{}\chead{}\rhead{}
\lfoot{}\cfoot{\thepage}\rfoot{}

%%% SECTION TITLE APPEARANCE
\usepackage{sectsty}
\allsectionsfont{\sffamily\mdseries\upshape} % (See the fntguide.pdf for font help)
% (This matches ConTeXt defaults)

%%% ToC (table of contents) APPEARANCE
\usepackage[nottoc,notlof,notlot]{tocbibind} % Put the bibliography in the ToC
\usepackage[titles,subfigure]{tocloft} % Alter the style of the Table of Contents
\renewcommand{\cftsecfont}{\rmfamily\mdseries\upshape}
\renewcommand{\cftsecpagefont}{\rmfamily\mdseries\upshape} % No bold!

%%% END Article customizations

%%% The "real" document content comes below...

\begin{document}
\section{System setup}
 In order to minimize the amount of factors that interfere in the tests we consider having at least three similar state of the art computers connected to a high-speed network essential.
It might not be of the greatest importance that the computers are state of the art but in order to not reach a hardware ceiling while testing, such machines are recommended. Whats most important is that one computer is designated to run ESBs, one is designated to generate traffic(called Client) while the others are simple servers responding to the traffic generated (called Web service). 

This separating and designating of roles to machines minimizes different hardware affecting test results as the same machines perform the same roles in all tests and the only thing changed is the ESB. 

It also means that if other machines are used in other tests the data produced can be compared to ours and as such validate the data or identify faults in the tests. This validation can be in two ways. First is to run the same software versions and compare the results. They should be similar deviating only in magnitude. The second way is if using a newer or old software version the values when put in a graph will either have the same shape or show areas in which performance has changed, if it is the same shape but the magnitude is higher then that is most likely caused by faster machines being used and vice versa if its the same magnitude except in certain areas then that shows an improvement in the software.

\subsection{Measurements}
Whats interesting to see here is the response time and throughput on the client. Those are the most interesting numbers too know in a real world application as those are extremely noticable to an end user/system. Monitoring the CPU and RAM usage on the ESB and Web service machines is somewhat important so that the tests are ruined by some hardware limitation. We will also use the CPU metrics from the ESB machine inorder to calculate relative performance on each ESB.

\begin{tabular}{| c | l |}
	\hline
	\multicolumn{2}{|c|}{Metrics captured} \\
	\hline
	Client & Response time and throughput \\ \hline
	ESB & CPU and RAM \\ \hline
	Web service &  CPU \\ \hline
\end{tabular}

\section{Test walktrough}
The tests described below are aimed at measuring the very basic roles of any ESB. They are very simple with the specific goal of producing a baseline for future tests.

All tests will be performed ten times and the collected data will then be statistically analyzed.
\\
------------------------ \\
TODO \\
Måste titta på hur vi ska analysera resultaten, typ som "Student’s Paired T-Test"

\subsection{Pure throughput}
The ESB will in this test not manipulate any data but instead will forward all requests as fast as possible. 
A comparative test can be performed here as well and that is to not use the ESB which will show what performance impact the ESB adds. 


\subsection{Routing}
In this test the ESB wil,l depending on the context of an incoming request from the Client, send the request to an appropriate web service which will append some data and return the request to the ESB which will send it to the Client.

This represents the ability to have several system behind an ESB all showing the same front to the outside.

\subsection{Message transformation}
The ESB will convert an incoming request to a different format and send it to a web service which will append some data and return the request to the ESB which will transform it back into the format the Client originally sent it. This second transform is meant to mimic how a real world application works, you do not send and receive using different protocols unless you're really forced too. 

\end{document}
