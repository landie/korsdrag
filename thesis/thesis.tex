\documentclass{llncs}

\usepackage[utf8x]{inputenc}
\usepackage{graphicx}
\usepackage{ctable}
\usepackage{tabularx}
\usepackage{subfig}


\title{Performance measurements on enterprise service bus}
\author{Joakim Olsson \inst{1} \and Johan Liljegren\inst{1}}

\institute{
	Blekinge Institute of Technology \\
	\email{laggmonkei@gmail.com}, \email{datanizze@gmail.com}
}

\hyphenation{}

\begin{document}
\maketitle
\begin{abstract}
Hola!
\end{abstract}

\section{Introduction}
% Introduction sets focus for thesis and make reader motivated to continue reading. to do that you have to clearly explain what's being investigated, why it is relevant and what value the thesis gives to the world.
% kan säkert slänga lite av detta i abstract sen...
The magic world of integration is upon us, it has long been a great part of the industry but never taken quite seriously.

In later years, in the beginning/middle of the 21st century to be more exact this problem, known as ``spaghetti code'' has taken a step too far, giving spark to a new subset of EAI (Enterprise Application Integration) and SOA  (Software As Service), namely the ESB (Enterprise Service Bus).
This new way of thinking wants to rid the world of the spaghetti code by placing the ESB between all connected platforms and act as a mediator - broadly termed. 
This rids the end to end connection between every application resulting in the applications (in a perfect world) would only be connected to the ESB with no other end-to-end integration.
% It should therefore contain:

% - research focus: what are you investigating
\subsection{Research focus}
The research focus lies on testing a number of ESB solutions' performance in the most essntial tasks a ESB provides. The expections on the outcome is to give a clear overview of the most popular ESBs' performance at the latest versions currently available.

% - rationale: why are you doing this
\subsection{Rationale}
When choosing an ESB to use a lot of parameters needs to be filled, ease of use, documentation, features and performance is a couple of things needed to make an educated decision what to use. This thesis focuses on performance since there are no current and unbiased resources about the most popular ESBs' performance.

% - problem definition: its nature and scope, why this problem? why is it important? (review relevant literature)
\subsection{Problem Definition}
The nature of this thesis is to clearly define tests and results of what currenct ESBs have to offer in the area of performance. This was chosen since the performance of an ESB plays a mayor role in perceived overall value of an esb since its job is broad and it handles just about every call between integrated platforms.

% - Aims: briefly and clearly
\subsection{Aims}
The Aims of this thesis is to get a grip of currenct ESBs performance and to evaluate what already exists and how relevant that is

% - method of investigation: why this method?
\subsection{Investigation method}
The chosen method of investigation is to create an environment where tests can be performed, singling out performance differences in the different ESB solution offerings.

% - expected outcome: value and intended audience
\subsection{Expected outcome}
The expected outcome of this thesis is to give a defined testbed with performance results from a controlled test environment for the involved ESB solutions.

% - limitations (avgränsningar)
%\subsection{Limitations}

% - thesis overview: what is presented in the following chapters and in what order
\subsection{Thesis Overview}
TODO: easier to do when we have some content to descibe.
% Vi får väl fixa här när vi skrivit lite mer

\section{Background}
\label{sec:background}

\section{Research question and research design}
\label{sec:method}
\section{Literature review results}
\label{sec:litrev}

\subsection{Academia}
\subsection{Industry}

\section{Performance test}
\subsection{What is a performance test?}
\subsection{Test walktrough}
	\subsection{Mule ESB}
	\subsection{WSO2}
	\subsection{Servicemix}
	\subsection{OpenESB}
	\subsection{UltraESB}

\section{Data synthesis and answer to research questions}
\section{Conclusion}
\end{document}
