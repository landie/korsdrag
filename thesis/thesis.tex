\documentclass{llncs}

\usepackage[utf8x]{inputenc}
\usepackage{graphicx}
\usepackage{ctable}
\usepackage{tabularx}
\usepackage{subfig}


\title{Performance measurements on enterprise service bus}
\author{Joakim Olsson \inst{1} \and Johan Liljegren\inst{1}}

\institute{
	Blekinge Institute of Technology \\
	\email{laggmonkei@gmail.com}, \email{datanizze@gmail.com}
}

\hyphenation{}

\begin{document}
\maketitle
\begin{abstract}
Hola!
\end{abstract}

\section{Introduction}
% Introduction sets focus for thesis and make reader motivated to continue reading. to do that you have to clearly explain what's being investigated, why it is relevant and what value the thesis gives to the world.
% kan säkert slänga lite av detta i abstract sen...
The magic world of integration is upon us, it has long been a great part of the industry but never taken quite seriously.

In later years, in the beginning/middle of the 21st century to be more exact this problem, known as ``spaghetti code'' has taken a step too far, giving spark to a new subset of EAI (Enterprise Application Integration) and SOA  (Software As Service), namely the ESB (Enterprise Service Bus).
This new way of thinking wants to rid the world of the spaghetti code by placing the ESB between all connected platforms and act as a mediator - broadly termed. 
This rids the end to end connection between every application resulting in the applications (in a perfect world) would only be connected to the ESB with no other end-to-end integration.
% It should therefore contain:

% - research focus: what are you investigating
\subsection{Research focus}
The research focus lies on testing a number of ESB solutions' performance in the most essntial tasks a ESB provides. The expections on the outcome is to give a clear overview of the most popular ESBs' performance at the latest versions currently available.

% - rationale: why are you doing this
\subsection{Rationale}
When choosing an ESB to use a lot of parameters needs to be filled, ease of use, documentation, features and performance is a couple of things needed to make an educated decision what to use. This thesis focuses on performance since there are no current and unbiased resources about the most popular ESBs' performance.

% - problem definition: its nature and scope, why this problem? why is it important? (review relevant literature)
\subsection{Problem Definition}
The nature of this thesis is to clearly define tests and results of what currenct ESBs have to offer in the area of performance. This was chosen since the performance of an ESB plays a mayor role in perceived overall value of an esb since its job is broad and it handles just about every call between integrated platforms.

% - Aims: briefly and clearly
\subsection{Aims}
The Aims of this thesis is to get a grip of currenct ESBs performance and to evaluate what already exists and how relevant that is

% - method of investigation: why this method?
\subsection{Investigation method}
The chosen method of investigation is to create an environment where tests can be performed, singling out performance differences in the different ESB solution offerings.

% - expected outcome: value and intended audience
\subsection{Expected outcome}
The expected outcome of this thesis is to give a defined testbed with performance results from a controlled test environment for the involved ESB solutions.

% - limitations (avgränsningar)
%\subsection{Limitations}

% - thesis overview: what is presented in the following chapters and in what order
\subsection{Thesis Overview}
TODO: easier to do when we have some content to descibe.
% Vi får väl fixa här när vi skrivit lite mer

\section{Background}
\label{sec:background}

\section{Research question and research design}
\label{sec:method}
\section{Literature review results}
\label{sec:litrev}
% - Present the papers that you have identified as your information source. Describe them, what type of papers they are, from what period; what do they have in common etc.


"Enterprise Service Bus" by Falko Menge is a paper that explains the fundamentals of an ESB as well as introduces Mule with an example. Published in 2007 its example has become obsolete and outdated however the fundamentals still hold true as seen by later papers such as "Research of Enterprise Application Integration Based-on ESB" by Jieming Wu and Xiaoli Tao as well as "Integration of Distributed Enterprise Applications: A Survey" by Wu He and Li Da Xu are papers that focus on describing the history and evolution of software integration.  They were published in 2010 and give an in depth view on how an ESB  operates.


"An Interoperability  Study  ofESB for C4I  Systems " by Abdullah Alghamdi, Muhammad Nasir, Iftikhar Ahmad and Khalid A. Nafjan and  "Adopting and Evaluating Service Oriented Architecture in Industry" by Khalid Adam Nasr, Hans-Gerhard Gross and Arie van Deursen are papers showing the importance of ESBs in the modern world. Published in 2010 they represent a modern necessity for ESBs in industry.


"Service-Oriented Performance Modeling the MULE Enterprise Service Bus (ESB) Loan Broker Application " by Paul Brebner is a paper from 2009 going trough how the author builds a integration solution and tests it, however the author tests the entire solution which isn't general enough for what we consider a performance test of an ESB. The author does however discuss some aspects of testing that are vital such as how and where to measure.

"Evaluating Open Source Enterprise Service Bus" by F. J. García-Jiménez and M. A. Martínez-Carreras, A. F. is the first paper found that performs a performance test however the test is limited in variation and magnitude. The paper was published in 2010 which makes the test outdated since all ESBs in the test has received major updates since 2010.

% - Present your approach for analyzing selected papers in order to find concepts/topics/data/events/experiences that are relevant for answering your research questions
The above papers has as per the literature review design been first identified by its abstract and then read in detail. 

% - Present and describe the findings from your literature survey and their relation to your research questions

Our literature review shows a very clear lack of performance tests conducted by academia in a organized and impartial way.
% - Discuss your findings and conclude literature survey results.


\subsection{Academia}
\subsection{Industry}

\section{Performance test}
\subsection{What is a performance test?}
\subsection{Test walktrough}
	\subsection{Mule ESB}
	\subsection{WSO2}
	\subsection{Servicemix}
	\subsection{OpenESB}
	\subsection{UltraESB}

\section{Data synthesis and answer to research questions}
\section{Conclusion}

\bibliographystyle{ieeetr}
\bibliography{refs}
% {7-12 key references that are relevant for your particular project. The reference list should be complete with all expected information for all entries.}
\end{document}
