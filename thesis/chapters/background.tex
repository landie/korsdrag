\section{Background}
\label{sec:background}
% used to explain and present concepts and terms important to understand the thesis... allts� bakgrundsinfo om integration, esb, xml, xpath osv.
% Also used to further explain problem/issue that we want to investigate and motivate importance of by f.e. showing what are the gaps in the currently available body of knowledge and that you are intending to fill in some of those gaps with knowledge gathered from our research.

% should contain the following parts:
% detailed information on the background concepts necessary to understand and evaluate your results - aka ge tillr�ckligt med info f�r att n�gon som inte alls �r insatt s� de kan f�rst� sisisisisen.
% An overview of the related literature and state-of-the-art research in order to identify the gap and motivate the need for our research.

Enterprise Service Buses (ESB) provides a platform for integration, connecting existing platforms and products with familiar protocols \cite{falko07}.
Before integration became an integral part of developing systems these now integrated parts where operating on its own without any interaction with other platforms/products. 
This can be viewed as ``independent islands of computing''. 
The ESBs task is to bring communication between these islands thus integrating them.

ESBs are pivotal in software integration and are becoming extremely important in company software solutions and infrastructures, future and ancient alike \cite{fenner03}.

As such it is very important to accurately and repeatedly measure the performance of integration solutions available. 
There are currently several open source options available \cite{mehta11} and the current body of knowledge regarding their different performance aspects is severely lacking. 

\subsection{Core functionality}
There are several different views on what the core functionalities of an ESB are. The most comprehensive list we have found was from Jieming \cite{Jieming2010}. 


	\begin{description}
		\item[Location transparency] \hfill \\
			Seperates the service consumer from the service provider allowing the provider to centralize hardware while still providing regional services.
		\item[Transport protocol conversion] \hfill \\
			Understanding several communication standards allows for several system to be integrated. Especially old legacy systems can be connected to new modern systems.
		\item[Message transformation] \hfill \\
			Being able to translate between communication standards also helps with integrating different systems.
		\item[Message routing] \hfill \\ 
			Being able to send an incoming message to the right system is vital for integrating systems as they can be seperated by location, age, protocol, country and many more criterias.
		\item[Message enhancement] \hfill \\
			The ability to add information to incoming messages allow the ESB to connect services and send their combined message to other services, ultimately adding longevity to all services.
		\item[Security] \hfill \\
			The ability to handle security is vital as the ESBs purpose is to connect several systems and as such is often the first and last defence.
		\item[Monitoring and management] \hfill \\
			Efficiency is key and as such being able to monitor and manage an ESB is necessary
	\end{description}

From this list we can extract a shorter more concise list of core performance functions.

	\begin{description}
		\item[Direct proxy] \hfill \\
			The ability to transfer incoming messages to a specific service.
		\item[Routing] \hfill \\
			Being able to look at incoming requests and based on their specifik content forward them to different systems.
		\item[Mediation/Transformation] \hfill \\
			Translating between different inbound and outbound protocols.
	\end{description}

These three functionalities are the very basic fundamental building blocks that all other ESB functionality depends on. 
An even better explanation of these core functions is that they represent an end to end flow trough the ESB. This means that one could simply replace the ESB with another ESB and the flow should be implementable without any disturbance to any other systems.
Security features aswell as monitoring features are then layered ontop of this flow and as such they could be considered extras or ESB specific.  


% {2-3 paragraphs giving more detailed background. Should answer: What has been done by others in this area? What is our current knowledge?}
There has been some measurements and benchmarks done by the producers of the ESBs however these measurements have always been in favor of the ESB produced by the company making the measurements. 
Impartial performance measurement are especially lacking, there are several reports made by the producers of some solutions \cite{Perera07,mulevsjboss,mulevsglassfish,mulevsservicemix,mulesoft08}.
There has been some research done in the area \cite{Sanjay2011} however the number of ESB's tested are quite limited when compared to the most up to date list of popular ESBs \cite{mehta11}.

Most importantly they do not explicitly declare what versions they are using in their tests which makes their results hard to reproduce. 
Judging by the dates on which their paper was completed it is clear that all the ESBs that they use have received major upgrades which makes it even more important to produce new measurements.

% {1 paragraph detailing the gap in our current knowledge. Should answer: What is missing in our current knowledge? What is the main purpose of doing this project? Overall, what will we do?}
The main purpose of this thesis is to increase the knowledge of differences between modern open source ESBs. 
This will hopefully lead to it being easier to further develop these software products and make it easier for end users to compare and evaluate based on their specific needs.
