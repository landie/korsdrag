\section{Introduction}
% Introduction sets focus for thesis and make reader motivated to continue reading. to do that you have to clearly explain what's being investigated, why it is relevant and what value the thesis gives to the world.

Since the beginning of software engineering the industry have come a long way in developing applications and platforms. 
This progress has been greatly beneficial for everyday tasks and for fulfilling industrial needs. 
The progress has been staggering but in some areas the industry have had a harder time. 
The issue lies in connecting these applications and platforms, making them communicate with each others. This is where the Enterprise Service Bus (ESB) \cite{falko07} comes in. 
To address these integration issues terms like Enterprise Application Integration (EAI) \cite{Du2008} and Service Oriented Architecture (SOA) \cite{Abuosba2008} have been visualized in the past. 
The ESB is a ``next step'' in this direction, taking what's good from SOA and EAI to bring a more complete platform for integration.

Early manual integration between different platforms (often referred to as ``independent islands of computing'') was done by point to point integration, that is, for each new platform to integrate all existing platforms had to build separate interfaces to communicate with the new one. 
This is a very cumbersome way to do it resulting in messy integration ending up with a lot of dependencies between all nodes.
This type of integration is referred to as ``spaghetti code'' \cite{spaghetticode} since connections are made from everything to everything. 
This is where the ESB comes in. The ESB acts a mediator, translator and monitor amongst many other things. The main task for the ESB is to step in and take over communications between all involved platforms. 
This is done by changing the involved platforms so they connect to the ESB and only to the ESB \cite{Sanjay2011}. 
The main advantages of this is that the esb takes care of where to send data between the platforms making the platforms agnostic to what to send in regards to formatting/data structure and where, they just send it to the ESB and the ESB takes care of forwarding it to the right destination with the right data at the right place in the right language.

The research focus will be on how to write a test framework to make it easier for others to test different ESBs in a transparent and unbiased way. 
Another focus is to evaluate what is currently available in both academia and in industry to provide a review of what has been done and how is has been done.

When deciding what ESB to use a lot of parameters needs to be investigated, ease of use, documentation, features and performance are a couple of things needed to make an educated decision of what to use. 
This thesis focuses not on providing a list of which to use but to make it easier for others to make their own decision depending on their specific needs.

% - problem definition: its nature and scope, why this problem? why is it important? (review relevant literature)
The nature of this thesis is to clearly define what is available regarding ESB testing and how it could be done in a more reproducable way. 
This reproducability is achieved by uploading all source code to a repository on github \cite{github}.
Github makes future testing easy and reliable for testers and readers since there is a clearly visible history covering the entire source code.


% - method of investigation: why this method?
We will review literature available on the Internet in order to get a good understanding of the current body of knowledge and ascertain a need for increasing that body of knowledge. 
This also allows us to review how previous performance measurements and comparisons have been done in this field of software design which will ensure that our test framework does not deviate or alienate previous research as well as has a well anchored position in the current software design community.

% - expected outcome: value and intended audience
The expected outcome of this thesis is to provide a defined test framework with sample results from a controlled test environment and to give a full review of the current body of knowledge regarding ESB solutions.

This thesis is aimed for individuals or enterprises that are in the position that they are starting to look at acquiring a integration platform and as such are interested in knowing which is the best performing ESB.

% - thesis overview: what is presented in the following chapters and in what order
The thesis starts of with a literature review that examines the current body of knowledge in academia and industry, ending in a summary and conclusion regarding what tests this body contain.
After the literature review we present our version of a testing framework followed by the results gathered from a live run of said framework.
The whole thing is rounded up with a conclusion discussing the answers to the research questions asked, the contribution and proposed future work.
