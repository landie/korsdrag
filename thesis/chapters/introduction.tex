\section{Introduction} % TODO: add references
% Introduction sets focus for thesis and make reader motivated to continue reading. to do that you have to clearly explain what's being investigated, why it is relevant and what value the thesis gives to the world.

Since the dawn of software engineering we have come a long way in developing applications and platforms. The leaps forward has been greatly beneficial for everyday tasks but since more and more of these applications and platforms needs to communicate with each other it has become clear that the industry had not kept up with integrating platforms. Enter the Enterprise Service Bus (ESB). To address these integration issues terms like Enterprise Application Integration (EAI) and Service Oriented Architecture (SOA) have been visualized. The ESB is a ``next step'' in this direction, taking what's good from SOA and EAI to bring a complete platform for integration.

Where early manual integration between different platforms (often referred to as ``independent islands of computing'') was done by point to point integration, that is, for each new platform to integrate all existing platforms had to build separate interfaces to communicate with the new one. This is a very cumbersome way to do it resulting in messy integration with a lot of dependencies in the end. This type of integration is referred to as ``spaghetti code'' since connections are made from everything to everything. This is where the ESB comes in. The ESB acts a mediator, translator and monitor amongst many things. The main task for the ESB is to step in and take over communications between all involved platforms. This is done by changing the involved platforms so they connect to the ESB and only to the ESB \cite{Sanjay2011}. The main advantages of this is that the esb takes care of where to send data between the platforms making the platforms agnostic to what to send and where, they just send it to the ESB and the ESB takes care of forwarding it to the right destination with the right data at the right place in the right language.

The research focus will be on testing a number of ESB solutions' performance in the most essential tasks a ESB provides. The expectations of the outcome is to give a clear overview of the most popular ESBs' performance at the latest versions currently available.

When choosing an ESB to use a lot of parameters needs to be filled, ease of use, documentation, features and performance is a couple of things needed to make an educated decision what to use. This thesis focuses on performance since there are no current and unbiased resources about the most popular ESBs' performance.

% - problem definition: its nature and scope, why this problem? why is it important? (review relevant literature)
The nature of this thesis is to clearly define tests and results of what current ESBs have to offer in the area of performance. This was chosen since the performance of an ESB plays a mayor role in perceived overall value of an ESB since its job is broad and it handles just about every call between integrated platforms. This merges well with the aims of the thesis, which is to get a grip of current ESBs performance and to evaluate what already exists and how relevant that is

% - method of investigation: why this method?
The chosen method of investigation is to create an environment where tests can be performed, singling out performance differences in the different ESB solution offerings.
We will review literature available on the Internet in order to get a good understanding of the current body of knowledge and ascertain a need for increasing that body of knowledge. This also allows us to review how previous performance measurements have been done in this field of software design which will ensure that our test suite does not deviate or alienate previous research as well as has a somewhat well anchored position in the current software design community.


% - expected outcome: value and intended audience
The expected outcome of this thesis is to give a defined test bed with performance results from a controlled test environment for the involved ESB solutions.

% - limitations (avgr�nsningar)
%\subsection{Limitations}

% - thesis overview: what is presented in the following chapters and in what order
TODO: thesis overview: easier to do when we have some content to describe.
% Vi f�r v�l fixa h�r n�r vi skrivit lite mer