section{Research question and research design}
%Chapter 3 � Research questions and research design
The research questions are as follows: 
\begin{itemize}
	\item RQ1: What is the current knowledge of ESBs in academica?
	\item RQ2: What is the current knowledge of ESBs in the industry?
	\item RQ3: What are the components of a transparent and unbiased ESB solutions comparison framework.
\end{itemize}
%Present clearly how the research questions will be answered. Some of the questions will be answered through the literature review, some through the experiment/interview/survey and some through both.
RQ1-RQ2 will be answered through the literature review.
RQ3 will be researched through literature review and answered by delivering our own testing framework.

%Present the design of your literature survey:
%How did you conduct the literature search? 
The survey design consisted of searching the different search engines listed below with the listed search strings. \\

{\bf Search engines used:}
\begin{itemize}
	\item IEEE Xplore
	\item Google
	\item Google Scholar
	\item BTHs general search engine for academical documents
\end{itemize}

{\bf Search strings used:}
\begin{itemize}
	\item ESB
	\item Enterprise Service Bus
	\item Enterprise Application Integration
	\item Integration
	\item Integration measurement
	\item Integration comparison
	\item Mule esb
	\item WSO2
	\item Testing
\end{itemize}

During the literature survey abstracts were read and if the abstract suited our criteria 
(Evaluations of open source ESBs, performance measurements, ESB comparisons amongst others) 
a quick look through the actual content to see if the abstract did indeed sum up the content in a fair way. 
If then the quick view through the paper gave any value it was read more thoroughly and then added to the list of accepted papers to use in this thesis.


%Here you should describe in detail how the experiment will be conducted, which valuables will be measured, How the experiment environment will be set up and controlled. 
