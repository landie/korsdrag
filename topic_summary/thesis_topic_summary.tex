% !TEX TS-program = pdflatex
% !TEX encoding = UTF-8 Unicode

% This is a simple template for a LaTeX document using the "article" class.
% See "book", "report", "letter" for other types of document.

\documentclass[11pt]{article} % use larger type; default would be 10pt

\usepackage[utf8]{inputenc} % set input encoding (not needed with XeLaTeX)

%%% Examples of Article customizations
% These packages are optional, depending whether you want the features they provide.
% See the LaTeX Companion or other references for full information.

%%% PAGE DIMENSIONS
\usepackage{geometry} % to change the page dimensions
\geometry{a4paper} % or letterpaper (US) or a5paper or....
\geometry{margin=1in} % for example, change the margins to 2 inches all round
% \geometry{landscape} % set up the page for landscape
%   read geometry.pdf for detailed page layout information

\usepackage{graphicx} % support the \includegraphics command and options

% \usepackage[parfill]{parskip} % Activate to begin paragraphs with an empty line rather than an indent

%%% PACKAGES
\usepackage{booktabs} % for much better looking tables
\usepackage{array} % for better arrays (eg matrices) in maths
\usepackage{paralist} % very flexible & customisable lists (eg. enumerate/itemize, etc.)
\usepackage{verbatim} % adds environment for commenting out blocks of text & for better verbatim
\usepackage{subfig} % make it possible to include more than one captioned figure/table in a single float
% These packages are all incorporated in the memoir class to one degree or another...

%%% HEADERS & FOOTERS
\usepackage{fancyhdr} % This should be set AFTER setting up the page geometry
\pagestyle{fancy} % options: empty , plain , fancy
\renewcommand{\headrulewidth}{0pt} % customise the layout...
\lhead{}\chead{}\rhead{}
\lfoot{}\cfoot{\thepage}\rfoot{}

%%% SECTION TITLE APPEARANCE
\usepackage{sectsty}
\allsectionsfont{\sffamily\mdseries\upshape} % (See the fntguide.pdf for font help)
% (This matches ConTeXt defaults)

%%% ToC (table of contents) APPEARANCE
\usepackage[nottoc,notlof,notlot]{tocbibind} % Put the bibliography in the ToC
\usepackage[titles,subfigure]{tocloft} % Alter the style of the Table of Contents
\renewcommand{\cftsecfont}{\rmfamily\mdseries\upshape}
\renewcommand{\cftsecpagefont}{\rmfamily\mdseries\upshape} % No bold!

%%% END Article customizations

%%% The "real" document content comes below...


\begin{document}
\section{Group members}
Joakim Olsson laggmonkei@gmail.com \\
Johan Liljegren datanizze@gmail.com

\section{Thesis title}
Comparison between current open source Enterprise Service Busses

\section{Background}
Enterprise Service Busses[1] are pivotal in software integration and are becoming extremely important in future software solutions and companies. [2]
There are currently several open source options available [3] and the current body of knowledge regarding their different performance aspects is severely lacking.


\subsection{Focus area}
The differences between the currently available open source ESB’s.
Mainly focusing on the performance difference and the ability to change parts of the solution to different software written in different languages.

\subsection{Importance/problem}
All software needs to integrate with some other software at some point in its lifetime.
This puts certain demands on the structure of the software aswell as how to connect these
softwares. That is where the ESB has started too play a pivotal role, but untangling
decades of adhoc implementations and software architectures is a massive undertaking.

\subsection{Thesis goal}
Give a comparative view of the different open source ESB’s available focusing on their
technical capabilities such as language indepence and throughput differences.

\section{References}
1: http://kanagwa.com/assets/21/Esb1.pdf \\
2:  http://www.cs.ucl.ac.uk/staff/ucacwxe/lectures/3C05-02-03/aswe21-essay.pdf \\
3: http://www.toolsjournal.com/integrations-articles/item/224-11-of-best-opensource-esb-tools \\

\end{document}
